\documentclass[12pt,a4paper]{article}
\usepackage[utf8]{inputenc}
\usepackage[brazil]{babel}
\usepackage{graphicx}
\usepackage{hyperref}
\usepackage{listings}
\usepackage{xcolor}
\usepackage{geometry}
\geometry{a4paper, margin=2.5cm}

% Configuração de código
\lstset{
    basicstyle=\ttfamily\small,
    breaklines=true,
    frame=single,
    language=Python,
    showstringspaces=false,
    numbers=left,
    numberstyle=\tiny,
    commentstyle=\color{gray},
    keywordstyle=\color{blue},
    stringstyle=\color{orange}
}

\title{\textbf{Agentes Autônomos}}
\subtitle{Relatório da Atividade Extra}
\author{[Givanildo de Sousa Gramacho]}
\date{\today}

\begin{document}

\maketitle
\newpage

\tableofcontents
\newpage

\section{Framework Escolhido}

Para o desenvolvimento do agente autônomo de Análise Exploratória de Dados (EDA), foi escolhido o framework \textbf{LangChain 0.3.12} combinado com \textbf{LLMs de última geração}.

\subsection{Justificativa da Escolha}

\begin{itemize}
    \item \textbf{LangChain}: Framework especializado em construção de aplicações com LLMs, oferecendo:
    \begin{itemize}
        \item Sistema robusto de agentes com capacidade de raciocínio (ReAct pattern)
        \item Memória conversacional integrada (ConversationBufferMemory)
        \item Suporte nativo para múltiplos provedores de LLM
        \item Ferramentas (tools) extensíveis para análise de dados
    \end{itemize}
    
    \item \textbf{Google Gemini 2.0 Flash}: Modelo LLM principal escolhido por:
    \begin{itemize}
        \item Alta velocidade de resposta
        \item API gratuita com limite generoso
        \item Capacidade multimodal
        \item Excelente desempenho em tarefas analíticas
    \end{itemize}
    
    \item \textbf{Streamlit 1.41.1}: Interface web interativa com:
    \begin{itemize}
        \item Deploy simplificado na Streamlit Cloud
        \item Componentes nativos para upload de arquivos
        \item Atualização reativa da interface
    \end{itemize}
\end{itemize}

\subsection{Arquitetura do Sistema}

O sistema utiliza a arquitetura de \textbf{Agente Autônomo ReAct} (Reasoning + Acting):

\begin{enumerate}
    \item \textbf{Pensamento (Thought)}: O agente analisa a pergunta
    \item \textbf{Ação (Action)}: Seleciona a ferramenta apropriada
    \item \textbf{Entrada (Action Input)}: Define os parâmetros
    \item \textbf{Observação (Observation)}: Recebe o resultado
    \item \textbf{Resposta Final (Final Answer)}: Sintetiza a resposta
\end{enumerate}

\newpage
\section{Estruturação da Solução}

\subsection{Módulos Principais}

\subsubsection{1. Agent Module (agent.py)}

Responsável pela orquestração do agente com memória:

\begin{lstlisting}[language=Python, caption=Inicialização do Agente]
def build_agent():
    # Configuracao do LLM
    llm = ChatGoogleGenerativeAI(
        model="gemini-2.0-flash-exp",
        temperature=0
    )
    
    # Memoria conversacional
    memory = ConversationBufferMemory(
        memory_key="chat_history",
        return_messages=True
    )
    
    # Agente ReAct com memoria
    agent = initialize_agent(
        tools=TOOLS,
        llm=llm,
        agent=AgentType.ZERO_SHOT_REACT_DESCRIPTION,
        memory=memory,
        handle_parsing_errors=True,
        max_iterations=8
    )
    
    return agent, llm
\end{lstlisting}

\subsubsection{2. Tools Module (tools.py + tools\_refactored.py)}

Conjunto de 18 ferramentas especializadas:

\begin{table}[h]
\centering
\begin{tabular}{|l|l|p{5cm}|}
\hline
\textbf{\#} & \textbf{Ferramenta} & \textbf{Função} \\ \hline
1 & schema & Retorna tipos de dados das colunas \\ \hline
2 & dataset\_info & Informações completas do dataset \\ \hline
3 & missing & Identifica valores ausentes \\ \hline
4 & describe & Estatísticas descritivas \\ \hline
5 & histogram & Gera histogramas \\ \hline
6 & boxplot & Cria boxplots \\ \hline
7 & scatter & Gráficos de dispersão \\ \hline
8 & correlation & Matriz de correlação \\ \hline
9 & outliers & Detecção de outliers (IQR/Z-score) \\ \hline
10 & clustering & K-means clustering \\ \hline
11-18 & ... & Análises adicionais \\ \hline
\end{tabular}
\caption{Ferramentas de Análise Implementadas}
\end{table}

\newpage
\subsubsection{3. Memory System}

Sistema de memória persistente implementado com:

\begin{lstlisting}[language=Python, caption=Sistema de Memória]
# Memoria conversacional
memory = ConversationBufferMemory(
    memory_key="chat_history",
    return_messages=True
)

# Prompt com historico integrado
suffix = """
Chat History: {chat_history}
Question: {input}
Thought: {agent_scratchpad}
"""
\end{lstlisting}

\textbf{Vantagens:}
\begin{itemize}
    \item Referências a análises anteriores
    \item Contexto preservado entre perguntas
    \item Conclusões baseadas em todo o histórico
\end{itemize}

\subsection{Fluxo de Execução}

\begin{enumerate}
    \item Usuário faz upload do arquivo CSV
    \item Dataset é carregado em memória thread-safe
    \item Usuário faz pergunta em linguagem natural
    \item Agente analisa a pergunta e consulta memória
    \item Agente seleciona ferramenta apropriada
    \item Ferramenta processa dados e retorna resultado
    \item Agente formula resposta em linguagem natural
    \item Resposta é exibida + gráficos (se aplicável)
    \item Interação é salva na memória
\end{enumerate}

\newpage
\section{Perguntas e Respostas}

\subsection{Pergunta 1: Informações Gerais do Dataset}

\textbf{Pergunta:} "Mostre informações completas do dataset"

\textbf{Resposta do Agente:}
\begin{verbatim}
Dataset Creditcard.csv - Informações:

Shape: 284.807 linhas x 31 colunas

Colunas: Time, V1, V2, ..., V28, Amount, Class

Tipos de Dados:
- Numéricas: 31 colunas (todas)
- Categóricas: 0 colunas
- Datetime: 0 colunas

Memória Utilizada: 67.89 MB

Valores Ausentes: Nenhum

Duplicatas: 0 registros
\end{verbatim}

\newpage
\subsection{Pergunta 2: Medidas de Tendência Central}

\textbf{Pergunta:} "Calcule média, mediana e desvio padrão da coluna Amount"

\textbf{Resposta do Agente:}
\begin{verbatim}
Estatísticas da Coluna Amount:

Média: 88.35 unidades monetárias
Mediana: 22.00 unidades monetárias
Moda: 0.00 unidades monetárias
Desvio Padrão: 250.12
Variância: 62.558,78
Coeficiente de Variação: 283.07%

Interpretação:
A grande diferença entre média (88.35) e mediana (22.00) 
indica distribuição assimétrica com presença de outliers.
O alto CV (283%) confirma grande variabilidade nos valores.
\end{verbatim}

\newpage
\subsection{Pergunta 3: Visualização - Histograma (COM GRÁFICO)}

\textbf{Pergunta:} "Crie um histograma da coluna Amount com 50 bins"

\textbf{Resposta do Agente:}
\begin{verbatim}
Histograma criado com sucesso!

Arquivo: plots/hist-Amount-20250108-223045.png

Estatísticas:
- Bins: 50
- Contagem: 284.807 valores
- Mín: 0.00
- Máx: 25.691,16
- Média: 88.35
- Mediana: 22.00
\end{verbatim}

\textbf{Gráfico Gerado:}

\begin{figure}[h]
\centering
\includegraphics[width=0.8\textwidth]{plots/hist-Amount-20250108-223045.png}
\caption{Histograma da Variável Amount - Distribuição Assimétrica}
\label{fig:histogram}
\end{figure}

\textbf{Análise do Gráfico:}
\begin{itemize}
    \item Distribuição fortemente assimétrica à direita
    \item Maioria dos valores concentrados próximo a zero
    \item Presença de outliers com valores extremos
    \item Sugere necessidade de transformação logarítmica
\end{itemize}

\newpage
\subsection{Pergunta 4: Detecção de Outliers}

\textbf{Pergunta:} "Detecte outliers na coluna Amount usando método IQR"

\textbf{Resposta do Agente:}
\begin{verbatim}
Análise de Outliers - Método IQR:

Quartis:
- Q1 (25%): 5.60
- Q2 (50%): 22.00
- Q3 (75%): 77.16

IQR (Intervalo Interquartil): 71.56

Limites:
- Limite Inferior: -39.54
- Limite Superior: 184.50

Outliers Detectados:
- Quantidade: 7.741 registros
- Percentual: 2.72% do dataset
- Valor Mínimo (outlier): 184.51
- Valor Máximo (outlier): 25.691,16

Recomendação:
Outliers representam transações de alto valor.
Considerar análise separada deste segmento.
\end{verbatim}

\newpage
\section{Conclusões do Agente}

\subsection{Pergunta sobre Conclusões}

\textbf{Pergunta:} "Gere uma conclusão completa sobre todas as análises realizadas"

\textbf{Resposta do Agente (Conclusão Automática):}

\begin{quote}
\textbf{RELATÓRIO EXECUTIVO - ANÁLISE EXPLORATÓRIA DO DATASET CREDITCARD}

\textbf{1. Resumo Executivo}

Foi realizada análise exploratória completa do dataset de transações de cartão de crédito contendo 284.807 registros e 31 variáveis. O dataset apresenta alta qualidade (sem valores ausentes ou duplicatas) e características específicas para detecção de fraudes.

\textbf{2. Principais Descobertas}

\textit{2.1 Características Gerais:}
\begin{itemize}
    \item Dataset balanceado em termos de qualidade (0\% missing)
    \item 31 variáveis numéricas (V1-V28, Time, Amount, Class)
    \item Variáveis V1-V28 já transformadas por PCA para privacidade
    \item Memória: 67.89 MB
\end{itemize}

\textit{2.2 Distribuição de Valores (Amount):}
\begin{itemize}
    \item Distribuição altamente assimétrica
    \item Média (88.35) >> Mediana (22.00) = presença de outliers
    \item CV = 283\% indica alta variabilidade
    \item 2.72\% de outliers (7.741 transações de alto valor)
\end{itemize}

\textit{2.3 Balanceamento de Classes:}
\begin{itemize}
    \item Classe 0 (Normal): 99.83\% (284.315 transações)
    \item Classe 1 (Fraude): 0.17\% (492 transações)
    \item \textbf{Dataset severamente desbalanceado}
\end{itemize}

\textbf{3. Insights Técnicos}

\textit{3.1 Padrões Identificados:}
\begin{itemize}
    \item Correlações fracas entre variáveis PCA (por design)
    \item Outliers em Amount não necessariamente indicam fraude
    \item Distribuição temporal uniforme
\end{itemize}

\textit{3.2 Implicações para Modelagem:}
\begin{itemize}
    \item Necessidade de técnicas de balanceamento (SMOTE, undersampling)
    \item Métrica de avaliação: F1-Score ou AUC-ROC (não acurácia)
    \item Considerar ensemble methods para lidar com desbalanceamento
\end{itemize}

\textbf{4. Recomendações}

\textit{4.1 Pré-processamento:}
\begin{itemize}
    \item Normalizar coluna Amount (StandardScaler)
    \item Normalizar coluna Time ou converter para features cíclicas
    \item Balancear classes antes do treinamento
\end{itemize}

\textit{4.2 Próximas Análises:}
\begin{itemize}
    \item Análise de correlação entre V1-V28 e Class
    \item Feature engineering temporal (hora do dia, dia da semana)
    \item Análise de padrões sequenciais em fraudes
\end{itemize}

\textit{4.3 Modelagem Sugerida:}
\begin{itemize}
    \item Random Forest com balanceamento
    \item XGBoost com scale\_pos\_weight
    \item Isolation Forest para detecção de anomalias
    \item Redes neurais com weighted loss
\end{itemize}

\textbf{5. Observações Finais}

O dataset apresenta características típicas de problemas de detecção de fraude: alta dimensionalidade, desbalanceamento extremo e necessidade de métricas especializadas. A análise exploratória revelou padrões consistentes que podem ser explorados em modelos preditivos com tratamento adequado do desbalanceamento.
\end{quote}

\newpage
\section{Códigos Fonte}

\subsection{Link do Repositório GitHub}

Todo o código fonte está disponível em:

\begin{center}
\url{https://github.com/seu-usuario/agentes_engenheiro_dados}
\end{center}

\subsection{Estrutura de Arquivos}

\begin{verbatim}
agentes_engenheiro_dados/
├── src/
│   ├── app.py                  # Interface Streamlit
│   ├── agent.py                # Agente com memória
│   ├── tools.py                # Ferramentas base
│   ├── tools_refactored.py     # Ferramentas adicionais
│   ├── memory_store.py         # Sistema de memória
│   └── utils.py                # Funções auxiliares
├── data/                       # Datasets
├── plots/                      # Gráficos gerados
├── requirements.txt            # Dependências
├── .gitignore                  # Proteção de secrets
└── README.md                   # Documentação completa
\end{verbatim}

\subsection{Principais Tecnologias}

\begin{itemize}
    \item Python 3.10+
    \item LangChain 0.3.12
    \item Streamlit 1.41.1
    \item Google Gemini 2.0 Flash
    \item Pandas, Matplotlib, Seaborn, Scikit-learn
\end{itemize}

\newpage
\section{Link para Acesso ao Agente}

\subsection{Aplicação Online}

O agente está disponível publicamente em:

\begin{center}
\Large
\url{https://seu-app.streamlit.app}
\end{center}

\textit{(Substitua pela URL real após fazer deploy no Streamlit Cloud)}

\subsection{Como Acessar}

\begin{enumerate}
    \item Acesse o link acima
    \item Faça upload de um arquivo CSV
    \item Digite perguntas em linguagem natural
    \item Visualize respostas e gráficos gerados
    \item Clique em "Gerar Conclusão Final" para relatório completo
\end{enumerate}

\subsection{Instruções para Deploy}

Para fazer deploy no Streamlit Cloud:

\begin{enumerate}
    \item Acesse \url{https://share.streamlit.io}
    \item Faça login com GitHub
    \item Selecione o repositório: \texttt{agentes\_engenheiro\_dados}
    \item Arquivo principal: \texttt{src/app.py}
    \item Configure secrets (API Keys) em Advanced Settings
    \item Aguarde deploy (2-3 minutos)
\end{enumerate}

\newpage
\section{Segurança e Boas Práticas}

\subsection{Proteção de Chaves de API}

\textbf{Medidas Implementadas:}

\begin{itemize}
    \item Arquivo \texttt{.env} incluído no \texttt{.gitignore}
    \item Secrets configurados no Streamlit Cloud (não no código)
    \item Variáveis de ambiente carregadas via \texttt{python-dotenv}
    \item Documentação sem exposição de chaves reais
\end{itemize}

\textbf{Arquivo .gitignore:}

\begin{lstlisting}
# Protecao de secrets
.env
*.env
.env.local
secrets.toml
*.key
credentials.json

# API Keys
GOOGLE_API_KEY
OPENAI_API_KEY
LANGSMITH_API_KEY
\end{lstlisting}

\subsection{Verificação Antes do Commit}

\begin{lstlisting}[language=bash]
# Sempre verificar antes de commitar
git status  # .env NAO deve aparecer
git diff    # Revisar mudancas
git add .
git commit -m "Seu commit"
git push
\end{lstlisting}

\newpage
\section{Conclusão}

Este trabalho apresentou o desenvolvimento completo de um agente autônomo para análise exploratória de dados utilizando o framework LangChain integrado com Google Gemini.

\subsection{Objetivos Alcançados}

\begin{itemize}
    \item[\checkmark] Framework moderno e eficiente (LangChain + Gemini)
    \item[\checkmark] 18 ferramentas de análise implementadas
    \item[\checkmark] Sistema de memória conversacional funcionando
    \item[\checkmark] Interface web interativa (Streamlit)
    \item[\checkmark] Deploy em nuvem (Streamlit Cloud)
    \item[\checkmark] Código open source no GitHub
    \item[\checkmark] Documentação completa
    \item[\checkmark] Proteção adequada de secrets
\end{itemize}

\subsection{Diferenciais da Solução}

\begin{enumerate}
    \item \textbf{Memória Persistente}: Diferente de chatbots simples, o agente lembra de todas as análises anteriores
    \item \textbf{Multi-LLM}: Suporta OpenAI, Gemini e Ollama
    \item \textbf{Conclusões Inteligentes}: Gera relatórios executivos automaticamente
    \item \textbf{Thread-Safe}: Permite múltiplos usuários simultâneos
    \item \textbf{Extensível}: Fácil adicionar novas ferramentas
\end{enumerate}

\subsection{Trabalhos Futuros}

\begin{itemize}
    \item Adicionar suporte para múltiplos formatos (Excel, JSON)
    \item Implementar cache de resultados
    \item Dashboard interativo com Plotly
    \item Exportação de relatórios em PDF
    \item Integração com bancos de dados
\end{itemize}

\newpage
\section*{Referências}

\begin{thebibliography}{9}

\bibitem{langchain}
LangChain Development Team (2024).
\textit{LangChain Documentation}.
Disponível em: \url{https://python.langchain.com/docs/}

\bibitem{gemini}
Google DeepMind (2024).
\textit{Gemini API Documentation}.
Disponível em: \url{https://ai.google.dev/}

\bibitem{streamlit}
Streamlit Inc. (2024).
\textit{Streamlit Documentation}.
Disponível em: \url{https://docs.streamlit.io/}

\bibitem{creditcard}
Kaggle (2024).
\textit{Credit Card Fraud Detection Dataset}.
Disponível em: \url{https://www.kaggle.com/datasets/mlg-ulb/creditcardfraud}

\end{thebibliography}

\end{document}
